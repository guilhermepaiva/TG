\chapter{Combina��o de Classificadores}
Neste cap�tulo, ser� mostrado os principais algoritmos de combina��o de classificadores, Bagging e Boosting. Cada se��o aborda os conceitos b�sicos de cada algoritmo, o seu pseudo-c�digo e suas caracter�sticas principais, bem como as varia��es mais conhecidas. Ao final, de maneira mais detalhada, ser� analisado o algoritmo ICS-Bagging com foco nas poss�veis m�tricas de diversidade que este algoritmo pode usar. 

\section{Bagging}
\textit{Bagging} � um algoritmo \cite{bagging:1996} de combina��o de classificadores que foi desenvolvido com o intuito de melhor a acur�cia na tarefa de classifica��o. O principal conceito deste algoritmo � o \textit{bootstrap aggregation}, na qual o conjunto de treinamento para cada classificador � constru�do a partir de uma amostra escolhida aleatoriamente do conjunto de treinamento. Os classificadores s�o ent�o combinados e ao se apresentar uma nova inst�ncia, cada classificador fornece como resultado a classe na qual essa nova inst�ncia pertence. Normalmente, a classe que obteve mais votos, � dita como sendo a classe desta inst�ncia. 


\begin{algorithm}
\caption{Bagging}
\label{alg:bagging}
\begin{algorithmic}
\FORALL {inst�ncia $e_i$ em $T$}
		\STATE Aplique o KNN sobre $e_i$ utilizando $T$ como treinamento
		\IF {$e_i$ foi classificado erroneamente}
			\STATE salve $e_i$ em $L$
		\ENDIF
	\ENDFOR
	\STATE Remova de $T$ os elementos de $L$
	\RETURN $T$
\end{algorithmic}
\end{algorithm}